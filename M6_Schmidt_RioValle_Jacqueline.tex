% Tipo de documento, tipo de papel, tamaño de letra
\documentclass[a4paper,11pt]{article}

% -----------------------------------------------------

% PAQUETES

% Paquete para usar reglas de castellano
\usepackage[spanish]{babel}
% Paquete para usar codificación UTF-8
\usepackage[utf8]{inputenc}
% Paquete para cambiar los márgenes
\usepackage{anysize}
% Paquete para insertar hipervínculos
\usepackage{hyperref}
% Paquete para poner color a las tablas
\usepackage[table,xcdraw]{xcolor}
% Paquete para insetar imágenes
\usepackage{graphicx}
\usepackage{float}
% Paquete para usar varias columnas
\usepackage{multicol}
% Paquete para usar enumerados
\usepackage{enumerate}
% Paquete para fuentes matemáticas
\usepackage{dsfont}

% -----------------------------------------------------

% FORMATO

% Ruta de las imágenes
\graphicspath{ {img/} }
% Fijar márgenes
\marginsize{2cm}{2cm}{2cm}{2cm}

% -----------------------------------------------------

% PORTADA

\title{\Huge{\textbf{PROYECTO FINAL. GIT + LATEX}}}
\author{\LARGE{Jacqueline Schmidt Rio-Valle}}
\date{\today}

% -----------------------------------------------------

% Documento

\begin{document}

\maketitle	
	
\newpage	
	
\tableofcontents	

\newpage
	
\section{Resumen}

El proyecto se encuentra disponible en el siguiente repositorio de \textbf{Github}: \url{https://github.com/jschmidt64/proyecto_final}\\

El objetivo del proyecto final es practicar los aspectos aprendidos durante el curso, y comprobar que se han comprendido adecuadamente. El proyecto contempla tres aspectos:

\begin{enumerate}
	\item La creación de un \textbf{repositorio} público que cumpla ciertas características que se describen más abajo.
	\item Crear un documento en \textbf{LaTeX} del mismo tipo que se utiliza en las publicaciones científicas.
	\item Usar el \textbf{repositorio} creado para controlar y documentar la redacción de ese artículo.
\end{enumerate}

\textbf{\underline{Descripción del repositorio:}}\\

\begin{itemize}
	\item Para empezar, crearemos un repositorio en nuestra cuenta de \textbf{github} que se llame \textit{proyecto\_final}.
	
	\item En este repositorio vamos a llevar el control de un documento en \textbf{LaTeX} que crearemos y que se describe más abajo.
	
	\item Todos los ficheros necesarios (fuentes en tex, imágenes, archivos de bibliografía, etc) deben estar en el \textbf{repositorio}.
	
	\item Naturalmente, el documento debe poder compilarse sin problemas.//          
\end{itemize}

\textit{\underline{Ojo}}: No es suficiente con hacer un sólo envío al repositorio con los ficheros necesarios. Debe haber, al menos, una sucesión de cinco commits con sus descripciones, etc.\\

\textbf{\underline{Descripción del documento:}}\\

Se trata de escribir un trabajo original completo que se asemeje a un artículo científico (naturalmente, no tiene por qué ser real) y que incluya, al menos, los siguientes aspectos:

\begin{enumerate}[{Apartado} 1.]
	\item Título
	\item Autores
	\item Resumen
	\item Palabras clave
	\item Introducción
	\item Listado de participantes
	\item Imágenes y tablas
	\item Fórmulas
	\item Bibliografía
\end{enumerate}

\section{Palabras clave}

\begin{itemize}
	\item Git
	\item Github
	\item LaTeX
	\item Paquete
	\item Repositorio
	\item BibTex
\end{itemize}

\section{Introducción}

En lo que a \textbf{LaTeX} respecta, hemos trabajado para conocer las diversas opciones que nos ofrece, ya sea en formato de texto, tablas, listas, imágenes, fórmulas, bibliografía (usando \textbf{BibTex}).\\

Además, hemos trabajado con \textbf{git} y la plataforma \textbf{Github} para llevar un control de versiones con los cambios que hemos ido realizando a lo largo del desarrollo del proyecto.

\section{Listado de participantes}

\begin{minipage}{0.5\linewidth}
	\begin{itemize}
		\item Carlos Gallardo Blanco
		\item Rayan Ruiz Fuentes
		\item José Molina Pastor
		\item Alonso Leon Carmona
		\item Éric Domenech Sanchez
		\item Adam Riera Navarro
		\item Bruno Moreno Ibañez
		\item Adria Duran Ruiz
	\end{itemize}
\end{minipage}
\begin{minipage}{0.5\linewidth}
	\begin{itemize}
		\item María Alvarez Navarro
		\item Ángela Ortega Ramos
		\item Inés Pons Pascual
		\item Nora Ferrer Parra
		\item Diana Castro Castro
		\item Laia Mora Vega
		\item Jana Sala Calvo
		\item Cristina Lozano Diez
	\end{itemize}
\end{minipage}

\newpage

\section{Imágenes y tablas}

\subsection{Tablas}

Introducimos una tabla:\\

\begin{center}
	\begin{tabular}{|c|c|c|}
		\hline
		\rowcolor[HTML]{C0C0C0} 
		\textbf{Variables} & \textbf{Varones} & \textbf{Mujeres} \\ \hline
		\textbf{Edad}      & 28               & 32               \\ \hline
		\textbf{Altura}    & 182              & 168              \\ \hline
	\end{tabular}
\end{center}

\subsection{Imágenes}

Introducimos imágenes:

\begin{multicols}{2}
	
	\begin{figure}[H]
		\caption{Universidad de Granada}
		\centering
		\includegraphics[width=0.5\linewidth]{ugr}
	\end{figure}

	\begin{figure}[H]
		\caption{Facultad de Ciencias de la Salud}
		\centering
		\includegraphics[width=0.5\linewidth]{fcs}
	\end{figure}

\end{multicols}


\section{Fórmulas}

\subsection{Fórmulas en línea}

\begin{itemize}
	\item Teorema de Pitágoras: $$\sqrt{a^2 + b^2} = c$$
	\item Equivalencia entre masa y energía: $$E = mc^2$$
	\item Área de una esfera: $$4\pi r^2$$
	\item Sucesión de Fibonachi: $$x_1=1,\quad x_2=1,\quad x_n=x_{n-1}+x_{n-2}\;\;(n>2)$$
	
\end{itemize}

\subsection{Fórmulas usando 'equation'}

\begin{equation}
\label{miecuacion}
f(x)=\sqrt{g’(x)dx}+\pi \backslash x \in \mathds{C} - \{0\}
\end{equation}

\begin{equation}
\label{miecuacion}
g(x)=\frac{4x+7}{x} + 3 \backslash x \in \mathds{N} - \{0\}
\end{equation}

\newpage

\bibliography{M6_Schmidt_RioValle_Jacqueline}
\bibliographystyle{plain}

\nocite{*}

\end{document}
