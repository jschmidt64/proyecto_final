% Tipo de documento, tipo de papel, tamaño de letra
\documentclass[a4paper,11pt]{article}

% -----------------------------------------------------

% PAQUETES

% Paquete para usar reglas de castellano
\usepackage[spanish]{babel}
% Paquete para usar codificación UTF-8
\usepackage[utf8]{inputenc}
% Paquete para cambiar los márgenes
\usepackage{anysize}
% Paquete para insertar hipervínculos
\usepackage{hyperref}
% Paquete para poner color a las tablas
\usepackage[table,xcdraw]{xcolor}
% Paquete para insetar imágenes
\usepackage{graphicx}
\usepackage{float}
% Paquete para usar varias columnas
\usepackage{multicol}

% -----------------------------------------------------

% FORMATO

% Ruta de las imágenes
\graphicspath{ {img/} }
% Fijar márgenes
\marginsize{2cm}{2cm}{2cm}{2cm}

% -----------------------------------------------------

% PORTADA

\title{\Huge{\textbf{PROYECTO FINAL. GIT + LATEX}}}
\author{\LARGE{Jacqueline Schmidt Rio-Valle}}
\date{\today}

% -----------------------------------------------------

% Documento

\begin{document}

\maketitle	
	
\newpage	
	
\tableofcontents	

\newpage
	
\section{Resumen}

El proyecto se encuentra disponible en el siguiente repositorio de \textbf{Github}: \url{https://github.com/jschmidt64/proyecto_final}

\section{Palabras clave}

\begin{itemize}
	\item Git
	\item Github
	\item Latex
	\item Paquete
	\item Repositorio
	\item BibTex
	\item TeXstudio
\end{itemize}

\section{Introducción}

El objetivo del proyecto final es practicar los aspectos aprendidos durante el curso, y comprobar que se han comprendido adecuadamente.

El proyecto contempla tres aspectos:

\begin{enumerate}
	\item La creación de un \textbf{repositorio} público que cumpla ciertas características que se describen más abajo.
	\item Crear un documento en \textbf{LateX} del mismo tipo que se utiliza en las publicaciones científicas.
	\item Usar el \textbf{repositorio} creado para controlar y documentar la redacción de ese artículo.
\end{enumerate}

\newpage

\section{Listado de participantes}

\begin{minipage}{0.5\linewidth}
	\begin{itemize}
		\item Carlos Gallardo Blanco
		\item Rayan Ruiz Fuentes
		\item José Molina Pastor
		\item Alonso Leon Carmona
		\item Éric Domenech Sanchez
		\item Adam Riera Navarro
		\item Bruno Moreno Ibañez
		\item Adria Duran Ruiz
	\end{itemize}
\end{minipage}
\begin{minipage}{0.5\linewidth}
	\begin{itemize}
		\item María Alvarez Navarro
		\item Ángela Ortega Ramos
		\item Inés Pons Pascual
		\item Nora Ferrer Parra
		\item Diana Castro Castro
		\item Laia Mora Vega
		\item Jana Sala Calvo
		\item Cristina Lozano Diez
	\end{itemize}
\end{minipage}

\section{Imágenes y tablas}

Introducimos una tabla:\\

	\begin{tabular}{|c|c|c|}
		\hline
		\rowcolor[HTML]{C0C0C0} 
		\textbf{Variables} & \textbf{Varones} & \textbf{Mujeres} \\ \hline
		\textbf{Edad}      & 28               & 32               \\ \hline
		\textbf{Altura}    & 182              & 168              \\ \hline
	\end{tabular}

\vspace{0.5cm}

Introducimos imágenes:

\begin{multicols}{2}
	
	\begin{figure}[H]
		\caption{Logo de la Universidad de Granada}
		\centering
		\includegraphics[width=0.5\linewidth]{ugr}
	\end{figure}

	\begin{figure}[H]
		\caption{E-learning}
		\centering
		\includegraphics[width=0.5\linewidth]{e-learning}
	\end{figure}
	
\end{multicols}


\section{Fórmulas}

\bibliography{M6_Schmidt_RioValle_Jacqueline}
\bibliographystyle{plain}

\nocite{*}

\end{document}
